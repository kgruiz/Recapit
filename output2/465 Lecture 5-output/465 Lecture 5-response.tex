\documentclass[aspectratio=43]{beamer}

\usepackage{amsmath}
\usepackage{amsfonts}
\usepackage{amssymb}
\usepackage{amsthm}
\usepackage{tikz}
\usepackage{xcolor}
\usepackage{array}
\usepackage{enumitem}
\usepackage{tabularx}

\usetheme{Madrid}
\usecolortheme{default}

\setbeamertemplate{navigation symbols}{}

\setbeamertemplate{footline}{}

\title{}
\author{}
\date{}

\begin{document}
\begin{frame}
\begin{center}
Math 465: Introduction to Combinatorics

\vspace{1cm}
Roberta Shapiro

\vspace{1cm}
Quiz \#2: 30 minutes, open Friday--Saturday.

\vspace{0.3cm}
Homework \#2 will be due Monday evening.

\vspace{1cm}
These slides will be posted on Canvas.
\end{center}
\end{frame}

\begin{frame}
\frametitle{Signless Stirling numbers of the first kind}

Recall that $c(w)$ denotes the number of cycles in a permutation $w$.

\begin{block}{Definition}
The numbers
\[
c(n,k) = \#\{w \in S_n \mid c(w) = k\} \quad (1 \leq k \leq n)
\]
are called the \textcolor{blue}{signless Stirling numbers of the first kind}.
\end{block}

There is no explicit closed formula for $c(n, k)$.

\begin{block}{Corollary}
\[
\sum_{k=1}^{n} c(n,k)x^k = \sum_{w \in S_n} x^{c(w)} = x(x+1)(x+2)\cdots(x+n-1)
\]
\end{block}

\end{frame}

\begin{frame}
Triangle of the signless Stirling numbers of the first kind

The recurrence
\[
c(n, k) = c(n - 1, k - 1) + (n - 1)c(n - 1, k)
\]
can be used to recursively compute the Stirling numbers $c(n, k)$.

\centering
\setlength\fboxsep{10pt}
\fcolorbox{lightgray}{lightgray}{
\begin{minipage}{0.8\textwidth}
\centering
Triangle of the numbers $c(n,k)$ for $1 \leq k \leq n \leq 7$

\vspace{0.5em}
\begin{tabular}{cccccccccccc}
         &         &         &         &         &         &         &  1       &         &         &         &         \\
         &         &         &         &         &         & 1        &         & 1        &         &         &         \\
         &         &         &         &         & 2        &         & 3        &         & 1        &         &         \\
         &         &         &         & 6        &         & 11       &         & 6        &         & 1        &         \\
         &         &         & 24       &         & 50       &         & 35       &         & 10       &         & 1        \\
         &         & 120      &         & 274      &         & 225      &         & 85       &         & 15       &         & 1        \\
         & 720      &         & 1764     &         & 1624     &         & 735      &         & 175      &         & 21       &         & 1        \\
$\cdots$ &         & $\cdots$ &         & $\cdots$ &         & $\cdots$ &         & $\cdots$ &         & $\cdots$ &         & $\cdots$ &         & $\cdots$
\end{tabular}
\end{minipage}
}

\bigskip

The numbering of rows/columns starts with $n = 1$ (resp., $k = 1$).
\end{frame}

\begin{frame}
\frametitle{Recurrence for the signless Stirling numbers of the $1^{\text{st}}$ kind}

\begin{block}{Proposition}
\[c(n,k) = c(n-1, k-1) + (n-1)c(n-1, k)\]
\end{block}

\begin{block}{Proof}
The formula
\[\sum_{k=1}^n c(n,k)x^k = x(x+1)(x+2)\cdots(x+n-1)\]
implies the identity
\[\sum_{k=1}^n c(n,k)x^k = (x+n-1)\sum_{j=1}^{n-1} c(n-1,j)x^j.\]
Now compare the coefficients of $x^k$ on both sides.
\end{block}

\begin{block}{Exercise}
Give a combinatorial proof of this proposition.
\end{block}

\end{frame}

\begin{frame}
\frametitle{Expanding falling powers via ordinary powers}

\begin{block}{Corollary}
\begin{align*}
\sum_{k=1}^{n} s(n, k) x^k &= x(x - 1)(x - 2)\cdots(x - n + 1) \\
&= (x)_n \\
&= n! \binom{x}{n}
\end{align*}
\end{block}

\begin{block}{Example: $n = 3$}
\[
2x - 3x^2 + x^3 = x(x - 1)(x - 2) = (x)_3.
\]
\end{block}

Thus the Stirling numbers of the first kind arise when we express the falling powers in terms of ordinary powers.

\end{frame}

\begin{frame}
\frametitle{Stirling numbers of the second kind}

\begin{block}{Definition}
The \alert{Stirling number of the second kind}, denoted $S(n, k)$, counts partitions of an $n$-element set into $k$ nonempty subsets (\alert{blocks}).

The blocks are \underline{not} labeled. Alternative viewpoint: enumeration of equivalence relations on $\{1, \dots, n\}$ with $k$ equivalence classes.
\end{block}

\begin{block}{Example: $n = 4$}
\begin{tabular}{c|c|c|c}
$S(4, 1) = 1$ & $S(4, 2) = 7$ & $S(4, 3) = 6$ & $S(4, 4) = 1$ \\
\hline
\begin{tabular}{|c|} \hline 1234 \\ \hline \end{tabular} &
\begin{tabular}{|c|c|} \hline 12 & 34 \\ \hline 13 & 24 \\ \hline 14 & 23 \\ \hline  & 4 \\ \hline \end{tabular} &
\begin{tabular}{|c|c|} \hline 1 & 2 34 \\ \hline 1 & 3 24 \\ \hline 1 & 4 23 \\ \hline 2 & 3 14 \\ \hline 2 & 4 13 \\ \hline 3 & 4 12 \\ \hline \end{tabular} &
\begin{tabular}{|c|c|c|c|} \hline 1 & 2 & 3 & 4 \\ \hline \end{tabular}
\end{tabular}
\end{block}

\end{frame}

\begin{frame}
    Generating functions for the numbers $s(n, k)$
    \begin{align*}
        s(n,k) &= (-1)^{n-k}c(n,k) \\
        \sum_{k=1}^{n} c(n,k)x^k &= x(x+1)(x+2) \cdots (x+n-1) \\
        \sum_{k=1}^{n} s(n,k)x^k &= \sum_{k=1}^{n} (-1)^{n-k}c(n,k)x^k \\
        &= (-1)^n \sum_{k=1}^{n} c(n,k)(-x)^k \\
        &= (-1)^n (-x)(-x+1)(-x+2) \cdots (-x+n-1) \\
        &= x(x-1)(x-2) \cdots (x-n+1)
    \end{align*}
\end{frame}

\begin{frame}
\frametitle{Signed Stirling numbers of the first kind}
\begin{block}{Definition}
The numbers
\begin{equation*}
    s(n, k) = (-1)^{n-k}c(n, k)
\end{equation*}
are called the \textcolor{blue}{signed Stirling numbers of the first kind}.
\end{block}

\begin{block}{Signed Stirling numbers of the first kind $s(n, k)$ for $1 \leq k \leq n \leq 6$}
\begin{center}
\begin{tabular}{cccccccccccc}
 & & & & & & & 1 & & & & \\
 & & & & & & -1 & & 1 & & & \\
 & & & & & 2 & & -3 & & 1 & & \\
 & & & & -6 & & 11 & & -6 & & 1 & \\
 & & & 24 & & -50 & & 35 & & -10 & & 1 \\
 & -120 & & 274 & & -225 & & 85 & & -15 & & 1 \\
$\cdots$ & $\cdots$ & $\cdots$ & $\cdots$ & $\cdots$ & $\cdots$ & $\cdots$ & $\cdots$ & $\cdots$ & $\cdots$ & $\cdots$ & $\cdots$
\end{tabular}
\end{center}
\end{block}
\end{frame}

\begin{frame}
\frametitle{Table of the Stirling numbers of the second kind}

The recurrence
$$S(n, k) = S(n - 1, k - 1) + k\, S(n - 1, k)$$
can be used to compute the Stirling numbers of the second kind:

\begin{center}
\setlength\tabcolsep{5pt}
\begin{tabular}{ccccccccccccccccc}
\multicolumn{17}{c}{\textbf{Triangle of the numbers $S(n,k)$ for $1 \leq k \leq n \leq 7$}} \\ \hline
& & & & & & & & & & & & & & & & \\
& & & & & & & & & & & & & & & 1 & \\
& & & & & & & & & & & & & & 1 & & 1 \\
& & & & & & & & & & & & & 1 & & 3 & & 1 \\
& & & & & & & & & & & & 1 & & 7 & & 6 & & 1 \\
& & & & & & & & & & & 1 & & 15 & & 25 & & 10 & & 1 \\
& & & & & & & & & & 1 & & 31 & & 90 & & 65 & & 15 & & 1 \\
& & & & & & & & & 1 & & 63 & & 301 & & 350 & & 140 & & 21 & & 1 \\
& & & & & & & & $\dots$ & $\dots$ & $\dots$ & $\dots$ & $\dots$ & $\dots$ & $\dots$ & $\dots$ & $\dots$ & $\dots$ & $\dots$ & $\dots$ & $\dots$ \\
\end{tabular}
\end{center}

\end{frame}

\begin{frame}
\frametitle{Recurrence for the Stirling numbers of the second kind}

\begin{block}{Theorem}
\[S(n, k) = S(n - 1, k - 1) + k S(n - 1, k)\]
\end{block}

\begin{block}{Proof}
Each partition of $\{1, \dots, n\}$ into $k$ blocks is of one of two kinds:
\begin{itemize}
    \item either it has a block $\boxed{n}$ consisting solely of $n$, or
    \item it can be obtained by adding $n$ to one of the blocks of a (unique) partition of $\{1, \dots, n-1\}$ into $k$ blocks.
\end{itemize}
\end{block}

\end{frame}

\begin{frame}
\frametitle{Proof of the formula $x^n = \sum S(n,k) (x)_k$}
\end{frame}

\begin{frame}
\frametitle{Theorem}
\[
x^n = \sum_{k=1}^n S(n,k) (x)_k
\]
\end{frame}

\begin{frame}
\frametitle{Proof}
It suffices to prove this identity when $x$ is a positive integer [why?].
In that case,
\[
x^n = \text{number of ways to color } \{1, \dots, n\} \text{ in } x \text{ colors};
\]
\[
\sum_k S(n,k) (x)_k = \text{number of ways to split } \{1, \dots, n\} \text{ into blocks,}
\]
\[
\text{then color these blocks in distinct colors.}
\]
\end{frame}

\begin{frame}
\frametitle{Expanding ordinary powers in terms of falling powers}
\begin{block}{Theorem}
\[
x^n = \sum_{k=1}^n S(n, k) (x)_k = \sum_{k=1}^n S(n, k) k! \binom{x}{k}
\]
Thus, the Stirling numbers of the second kind arise when we expand a monomial $x^n$ as a linear combination of the falling powers of $x$ (or of binomial coefficients $\binom{x}{k}$)).
\end{block}

\begin{block}{Examples}
\[
x^2 = (x)_1 + (x)_2 = \binom{x}{1} + 2\binom{x}{2}
\]
\[
x^3 = (x)_1 + 3(x)_2 + (x)_3 = \binom{x}{1} + 6\binom{x}{2} + 6\binom{x}{3}.
\]
\end{block}

\end{frame}

\begin{frame}
\frametitle{Stirling numbers as entries of transition matrices}

Consider the $n$-dimensional vector space
\begin{center}
\{polynomials in $x$ of degree $\leq n$ with constant term 0\}.
\end{center}
This vector space has two distinguished bases:
\begin{itemize}
    \item the basis of monomials $(x, x^2, \dots, x^n)$;
    \item the basis of falling powers $((x)_1, (x)_2, \dots, (x)_n)$.
\end{itemize}
The Stirling numbers of the two kinds describe the transitions from one of these bases to another. Consequently, the (lower-triangular) matrices made of these numbers are inverse to each other.

\bigskip
\begin{block}{Example}
\[
\begin{bmatrix}
    1 & 0 & 0 \\
    -1 & 1 & 0 \\
    2 & -3 & 1
\end{bmatrix}
=
\begin{bmatrix}
    1 & 0 & 0 \\
    1 & 1 & 0 \\
    1 & 3 & 1
\end{bmatrix}^{-1}.
\]
\end{block}

\end{frame}

\begin{frame}
\frametitle{Adding binomial coefficients}

\begin{block}{Lemma}
\[
\sum_{j=0}^n \binom{j}{k} = \sum_{j=k}^n \binom{j}{k} = \binom{n+1}{k+1}
\]
\end{block}

\begin{center}
\begin{tabular}{ccccccccccccccccc}
&&&&&&&&&1&&&&&&&&\\
&&&&&&&&1&&1&&&&&&&\\
&&&&&&&1&&2&&1&&&&&&\\
&&&&&&1&&3&&\textcolor{cyan}{3}&&1&&&&&\\
&&&&&1&&4&&6&&4&&1&&&&\\
&&&&1&&5&&\textcolor{cyan}{10}&&\textcolor{cyan}{10}&&5&&1&&&\\
&&&1&&6&&\textcolor{cyan}{15}&&20&&15&&6&&1&&\\
&&1&&7&&21&&\textcolor{red}{35}&&35&&21&&7&&1&\\
&\multicolumn{17}{c}{\dots \dots \dots \dots \dots \dots \dots \dots \dots \dots \dots \dots \dots \dots \dots \dots \dots}
\end{tabular}
\end{center}

\end{frame}

\begin{frame}
\frametitle{Adding the squares}
\end{frame}

\begin{frame}
\frametitle{Proposition}
    \[ \sum_{j=1}^n j^2 = \frac{n(n+1)(2n+1)}{6} \]
\end{frame}

\begin{frame}
\frametitle{Proof}
\begin{align*}
    \sum_{j=1}^n j^2 &= \sum_{j=1}^n \left( \binom{j}{1} + 2\binom{j}{2} \right) \\
    &= \binom{n+1}{2} + 2\binom{n+1}{3} \\
    &= \frac{(n+1)n}{2} + 2\frac{(n+1)n(n-1)}{3!} \\
    &= \frac{(n+1)n}{2} + \frac{(n+1)n(n-1)}{3} \\
    &= \frac{n(n+1)(2n+1)}{6}
\end{align*}
\end{frame}

\begin{frame}
\frametitle{Summation formulas}

More generally, expressing a polynomial $f(j)$ in terms of binomial coefficients $\binom{j}{k}$ leads to a summation formula for $f$:

\begin{align*}
f(j) &= \sum_k a_k \binom{j}{k} \\
\Rightarrow \sum_{j=0}^n f(j) &= \sum_{j=0}^n \sum_k a_k \binom{j}{k} \\
&= \sum_k a_k \binom{n+1}{k+1}
\end{align*}

\end{frame}
\end{document}