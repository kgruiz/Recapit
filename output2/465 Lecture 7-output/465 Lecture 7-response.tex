\documentclass[aspectratio=43]{beamer}

\usepackage{amsmath}
\usepackage{amsfonts}
\usepackage{amssymb}
\usepackage{amsthm}
\usepackage{tikz}
\usepackage{xcolor}
\usepackage{array}
\usepackage{enumitem}
\usepackage{tabularx}

\usetheme{Madrid}
\usecolortheme{default}

\setbeamertemplate{navigation symbols}{}

\setbeamertemplate{footline}{}

\title{}
\author{}
\date{}

\begin{document}
\begin{frame}
\frametitle{Roots of polynomials}

\begin{block}{Definition}
We say that $q_0$ is a \textcolor{blue}{root of multiplicity $m$} of a polynomial $P(q)$ (or of the equation $P(q) = 0$) if
\[ P(q) = (q - q_0)^m R(q) \]
where $R(q)$ is a polynomial in $q$ such that $R(q_0) \neq 0$.
\end{block}

\begin{block}{Lemma [no proof here]}
For a polynomial $P(q)$ and a number $q_0$, the following are equivalent:
\begin{itemize}
    \item $q_0$ is a root of $P(q)$ of multiplicity $m$;
    \item $q_0$ is a root of $P(q), P'(q), P''(q), \dots, P^{(m-1)}(q)$ but not $P^{(m)}(q)$.
\end{itemize}
\end{block}

\begin{block}{Fundamental Theorem of Algebra [no proof here]}
Any polynomial $P(q)$ of degree $k > 0$ with complex coefficients has precisely $k$ complex roots, counted with multiplicities.
\end{block}

\end{frame}

\begin{frame}
\frametitle{New solutions for multiple roots (1)}

\begin{block}{Proposition}
Let $q_0 \neq 0$ be a root of multiplicity $\geq 2$ of the characteristic equation
\\
(**) $\displaystyle \sum_{i=0}^{k} a_i \, q_0^{k-i} = 0$.
\\
Then $h_n = n \, q_0^n$ is a solution of the linear recurrence (*).
\end{block}

\begin{block}{Proof}
We want to show that
$$ \sum_{i=0}^{k} a_i \cdot (n-i) \, q_0^{n-i} = 0. $$
Multiplying (**) by $q_0^{n-k}$, we see that $q_0$ is a multiple root of
$$ \sum_{i=0}^{k} a_i \, q_0^{n-i} = 0. $$
Therefore $q_0$ is a root of the equation
$$ \sum_{i=0}^{k} a_i \cdot (n - i) \, q_0^{n-i-1} = 0. $$
The claim follows.
\end{block}

\end{frame}

\begin{frame}
\frametitle{Linear recurrences with constant coefficients, general case}

Recall that we are interested in solving the linear recurrence
\begin{equation*}
    (*) \qquad \sum_{i=0}^{k} a_i\, h_{n-i} = 0
\end{equation*}
where $n \geq k$, $a_0=1$ and $a_k \neq 0$.

We studied the case when the characteristic equation
\begin{equation*}
    (**) \qquad \sum_{i=0}^{k} a_i\, q^{k-i} = 0
\end{equation*}
has distinct roots. What if it has multiple roots?

\end{frame}

\begin{frame}
\begin{center}
Math 465: Introduction to Combinatorics

\vspace{1cm}

Andrew Sack

\vspace{1cm}
\hrulefill

\vspace{0.5cm}

Homework \#3 will be due Monday evening.

\vspace{0.5cm}
\hrulefill

\vspace{0.5cm}

These slides will be posted on Canvas.
\end{center}
\end{frame}

\begin{frame}
\frametitle{New solutions for multiple roots (2)}

\begin{block}{Corollary}
Let $q_0 \neq 0$ be a multiple root of the characteristic equation (**).
Then both $h_n = q_0^n$ and $h_n = n q_0^n$ satisfy the linear recurrence (*).
\end{block}

\begin{block}{Proposition/Exercise}
Let $q_0$ be a root of multiplicity $m$ of the characteristic equation (**).
Then for any $\ell \in \{0, 1, \dots, m-1\}$, the sequence $h_n = n^\ell q_0^n$ satisfies (*).

As $q_0$ varies over all roots of (**), we obtain $k$ such solutions, by the Fundamental Theorem of Algebra.
\end{block}

\begin{block}{Proposition/Exercise}
These $k$ solutions are linearly independent. Consequently, they form a basis in the space of all solutions of (*).
\end{block}

\end{frame}

\begin{frame}
\frametitle{General solution in the case of multiple roots}
\begin{block}{Theorem}
Consider the linear recurrence
\begin{equation*}
(*) \qquad \sum_{i=0}^{k} a_i h_{n-i} = 0.
\end{equation*}
Let $q_1, \dots, q_r$ be the complex roots of its characteristic equation
\begin{equation*}
(**) \qquad \sum_{i=0}^{k} a_i q^{k-i} = 0,
\end{equation*}
with respective multiplicities $m_1, \dots, m_r$. Then the general solution of the recurrence $(*)$ is given by
\begin{equation*}
h_n = C_1(n)q_1^n + \dots + C_r(n)q_r^n
\end{equation*}
where each $C_i(n)$ is a polynomial in $n$ of degree $\deg(C_i) < m_i$.
\end{block}
\end{frame}

\begin{frame}
\frametitle{Non-homogeneous linear recurrences}
\begin{block}{Problem}
Let $a_1, \dots, a_k$ be as before. Given a sequence $(b_n) = (b_0, b_1, b_2, \dots)$, find the solution of a linear non-homogeneous recurrence
\begin{align*}
    (*') \quad h_n + a_1h_{n-1} + a_2h_{n-2} + \dots + a_kh_{n-k} = b_n \quad (n \geq k)
\end{align*}
satisfying prescribed initial conditions.
\end{block}

\begin{block}{Observation}
The difference of any two solutions of the recurrence $(*')$ satisfies the corresponding homogeneous recurrence $(*)$.

Hence it suffices to find one particular solution of $(*')$. The general solution of $(*')$ is then obtained by adding to this particular solution the general solution of $(*)$.

A particular solution of $(*')$ can often be found within the same class of sequences that $(b_n)$ belongs to. Thus, if $b_n$ is a polynomial in $n$, try to find $h_n$ that is also a polynomial. If $b_n = b_0r^n$, try $h_n = h_0r^n$, etc.
\end{block}

\end{frame}

\begin{frame}
\frametitle{Case of multiple roots: Example}

\begin{block}{Problem}
Solve the linear recurrence $h_n = 3h_{n-2} + 2h_{n-3}$ with the initial conditions $h_0 = 0, h_1 = 0, h_2 = 1$.
\end{block}

\begin{center}
\begin{tabular}{||c||c|c|c|c|c|c|c|c|c|c||}
 \hline
 $n$ & 0 & 1 & 2 & 3 & 4 & 5 & 6 & 7 & 8 & 9 \\
 \hline
 $h_n$ & 0 & 0 & 1 & 0 & 3 & 2 & 9 & 12 & 31 & 54 \\
 \hline
\end{tabular}
\end{center}

\begin{block}{Solution}
The characteristic equation
\[q^3 - 3q - 2 = (q+1)^2(q-2) = 0\]
has roots $-1, -1, 2$. Hence $h_n = (c_1 + c_2n)(-1)^n + c_3 \cdot 2^n$. Now
\[
\begin{cases}
c_1 + c_3 = 0 \\
-c_1 - c_2 + 2c_3 = 0 \\
c_1 + 2c_2 + 4c_3 = 1
\end{cases}
\implies
\begin{cases}
c_1 = -\frac{1}{9} \\
c_2 = \frac{1}{3} \\
c_3 = \frac{1}{9}
\end{cases}
\implies
h_n = \frac{(3n-1)(-1)^n + 2^n}{9}
\]
\end{block}

\end{frame}

\begin{frame}
\frametitle{Non-homogeneous linear recurrences: an example}

\textbf{Problem}

The sequence $(h_n) = (1, 1, 4, 5, 14, 23, 52, 97, \dots)$ is defined by
\[h_n = h_{n-1} + 2h_{n-2} + (-1)^n,\]
with the initial conditions $h_0 = h_1 = 1$. Find a formula for $h_n$.

\bigskip

\textbf{Solution}

The characteristic equation is $q^2 - q - 2 = 0$. Its roots are 2 and $-1$.

In this example, $b_n = (-1)^n$. Since $-1$ is a root of the characteristic equation, there is no solution of the form $h_n = c \cdot (-1)^n$. Let us try the next best thing, namely $h_n = cn(-1)^n$:

\begin{align*}
h_n - h_{n-1} - 2h_{n-2} &= c\left[n(-1)^n - (n-1)(-1)^{n-1} - 2(n-2)(-1)^{n-2}\right] \\
&= c(-1)^n [n + n - 1 - 2n + 4] \\
&= 3c(-1)^n,
\end{align*}

so $c = \frac{1}{3}$ works! Thus $h_n = c_1 2^n + c_2(-1)^n + \frac{1}{3}n(-1)^n$. Fitting $c_1$ and $c_2$ to the initial conditions, we obtain $h_n = \frac{7}{9}2^n + \frac{2}{9}(-1)^n + \frac{1}{3}n(-1)^n$.

\end{frame}

\begin{frame}
\frametitle{Characterization of sequences given by polynomials (1)}

We next discuss an important application of the theory of (homogeneous) linear recurrences.

\bigskip

\begin{block}{Corollary [see next slide for the proof]}
The general solution $(h_n)$ of the linear recurrence
\[
\sum_{i=0}^{k} (-1)^i \binom{k}{i} h_{n-i} = 0
\]
is a polynomial in $n$ of degree $\leq k-1$.
\end{block}

\bigskip

\begin{block}{Example ($k=4$)}
The solutions of the linear recurrence
\[
h_n - 4h_{n-1} + 6h_{n-2} - 4h_{n-3} + h_{n-4} = 0
\]
are given by $h_n = an^3 + bn^2 + cn + d$.

\bigskip
Another example: $k = 2$ (arithmetic progressions).
\end{block}
\end{frame}

\begin{frame}
\frametitle{Characterization of sequences given by polynomials (4)}

\begin{block}{Theorem}
A sequence $h = (h_n)$ is given by a polynomial in $n$ of degree $\leq k-1$
if and only if all of its $k$'th differences vanish.
\end{block}

\begin{block}{Proof \#1}
The vanishing of the $k$'th differences is equivalent to the equation
\[
\sum_{i=0}^k (-1)^i \binom{k}{i} h_{n-i} = 0.
\]
Solutions of this equations are given by polynomials of degree $< k$.
\end{block}

\begin{block}{Proof \#2}
It suffices to show [why?] that $h_n$ is a polynomial in $n$ of degree $d$
if and only if $h_{n+1} - h_n$ is a polynomial in $n$ of degree $d-1$.
One direction follows by observing that $(n+1)^a - n^a$ is a polynomial of
degree $a-1$. For the opposite direction, use summation formulas.
\end{block}

\end{frame}

\begin{frame}
\frametitle{Entries of a difference table}

\begin{block}{Proposition}
The $k$'th differences of a sequence $(h_n)$ are given by
\[
\sum_{i=0}^{k} (-1)^i \binom{k}{i} h_{n-i},
\]
for $n \geq k$.
\end{block}

\begin{block}{Proof}
\begin{align*}
&h_0 \qquad h_1 \qquad h_2 \qquad h_3 \qquad h_4 \\
&h_1 - h_0 \qquad h_2 - h_1 \qquad h_3 - h_2 \qquad h_4 - h_3 \\
&h_2 - 2h_1 + h_0 \qquad h_3 - 2h_2 + h_1 \qquad h_4 - 2h_3 + h_2 \\
&h_3 - 3h_2 + 3h_1 - h_0 \qquad h_4 - 3h_3 + 3h_2 - h_1
\end{align*}
\end{block}

\end{frame}

\begin{frame}
\frametitle{Characterization of sequences given by polynomials (3)}

\begin{block}{Theorem}
A sequence $h = (h_n)$ is given by a polynomial in $n$ of degree $\leq k-1$
if and only if all of its $k$'th differences vanish.
\end{block}

\begin{block}{Example: $h_n = n^3$}
\begin{center}
\begin{tabular}{ccccccccc}
0 & 1 & 8 & 27 & 64 & 125 & 216 & 343 & $\dots$ \\
 & 1 & 7 & 19 & 37 & 61 & 91 & 127 & $\dots$ \\
 & & 6 & 12 & 18 & 24 & 30 & 36 & $\dots$ \\
 & & & 6 & 6 & 6 & 6 & $\dots$ \\
 & & & & 0 & 0 & 0 & $\dots$ \\
 & & & & & 0 & 0 & $\dots$
\end{tabular}
\end{center}
\end{block}

\end{frame}

\begin{frame}
\frametitle{Difference tables}

For an infinite sequence $(h_n)$, the associated \textcolor{blue}{difference sequence} is the sequence $(h_{n+1} - h_n)$:

\begin{center}
\begin{tabular}{cccccccc}
& $h_0$ & & $h_1$ & & $h_2$ & & $h_3$ & & $h_4$ & & $\dots$ \\
& & $h_1 - h_0$ & & $h_2 - h_1$ & & $h_3 - h_2$ & & $h_4 - h_3$ & & $\dots$ &
\end{tabular}
\end{center}

Iterating, we obtain a \textcolor{blue}{difference table}, shown below for $h_n = \binom{n}{3}$:

\begin{center}
\begin{tabular}{ccccccccc}
$0$ & $0$ & $0$ & $1$ & $4$ & $10$ & $20$ & $35$ & $\dots$ \\
& $0$ & $0$ & $1$ & $3$ & $6$ & $10$ & $15$ & $\dots$ \\
& & $0$ & $1$ & $2$ & $3$ & $4$ & $5$ & $\dots$ \\
& & & $1$ & $1$ & $1$ & $1$ & $\dots$ &  \\
& & & & $0$ & $0$ & $0$ & $\dots$ & \\
& & & & & $0$ & $0$ & $\dots$ &
\end{tabular}
\end{center}

\end{frame}

\begin{frame}
\frametitle{Characterization of sequences given by polynomials (2)}

\begin{block}{Corollary}
The general solution $(h_n)$ of the linear recurrence
\[
\sum_{i=0}^{k} (-1)^i \binom{k}{i} h_{n-i} = 0
\]
is a polynomial in $n$ of degree $\leq k-1$.
\end{block}

\begin{block}{Proof}
In this case, the characteristic equation is
\[
\sum_{i=0}^{k} (-1)^i \binom{k}{i} q^{k-i} = (q-1)^k = 0.
\]
Hence the general solution is $h_n = C(n) \cdot 1^n$ where $C(n)$ is a polynomial of degree $< k$.
\end{block}

\end{frame}
\end{document}