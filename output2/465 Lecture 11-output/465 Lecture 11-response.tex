\documentclass[aspectratio=43]{beamer}

\usepackage{amsmath}
\usepackage{amsfonts}
\usepackage{amssymb}
\usepackage{amsthm}
\usepackage{tikz}
\usepackage{xcolor}
\usepackage{array}
\usepackage{enumitem}
\usepackage{tabularx}

\usetheme{Madrid}
\usecolortheme{default}

\setbeamertemplate{navigation symbols}{}

\setbeamertemplate{footline}{}

\title{}
\author{}
\date{}

\begin{document}
\begin{frame}
\frametitle{Submultisets of a multiset}

\begin{block}{Problem}
I have 2 pennies, 3 nickels, 4 dimes, and 5 quarters. In how many ways can I select 10 coins? Coins of equal value are indistinguishable.
\end{block}

\begin{block}{Solution}
We want to count nonnegative integer solutions of the equation
$$x_1 + x_2 + x_3 + x_4 = 10$$
satisfying $x_1 \leq 2, x_2 \leq 3, x_3 \leq 4, x_4 \leq 5$. Let us set
$$S = \{ (x_1, \dots, x_4) \mid x_1 + \dots + x_4 = 10, \; x_1, \dots, x_4 \geq 0 \}$$
$$A_1 = \{ (x_1, \dots, x_4) \in S \mid x_1 \geq 3 \}$$
$$A_2 = \{ (x_1, \dots, x_4) \in S \mid x_2 \geq 4 \}$$
$$A_3 = \{ (x_1, \dots, x_4) \in S \mid x_3 \geq 5 \}$$
$$A_4 = \{ (x_1, \dots, x_4) \in S \mid x_4 \geq 6 \}$$
We are interested in determining $|S - (A_1 \cup A_2 \cup A_3 \cup A_4)|$.
\end{block}

\end{frame}

\begin{frame}
\frametitle{Compositions with bounded parts}

\textbf{Problem}
\vspace{0.1cm}

Count weak compositions of 10 with 10 parts all of which are $\leq 2$.

\vspace{0.3cm}
\textbf{Solution}
\vspace{0.1cm}

\begin{align*}
S &= \{x_1 + \cdots + x_{10} = 10, \ x_1, \dots, x_{10} \geq 0 \} \\
A_i &= \{(x_1, \dots, x_{10}) \in S \ | \ x_i \geq 3 \} \\
|S - (A_1 \cup \dots \cup A_{10})| &= ? \\
|S| &= |\{x_1 + \cdots + x_{10} = 10\}| = \binom{19}{9} \\
|A_i| &= |\{y_1 + \cdots + y_{10} = 7\}| = \binom{16}{9} \\
|A_i \cap A_j| &= |\{z_1 + \cdots + z_{10} = 4\}| = \binom{13}{9} \\
|A_i \cap A_j \cap A_k| &= |\{u_1 + \cdots + u_{10} = 1\}| = \binom{10}{9} \\
|S - (A_1 \cup \dots \cup A_{10})| &= \binom{19}{9} - 10 \cdot \binom{16}{9} + \binom{10}{2}\binom{13}{9} - \binom{10}{3} \binom{10}{9} = 8953
\end{align*}

\end{frame}

\begin{frame}
\frametitle{Euler's totient function}
\end{frame}

\begin{frame}
\begin{block}{Theorem [L. Euler]}
Let $\{p_1, \dots, p_r\}$ be the set of distinct prime divisors of $n$. Then
\[
\phi(n) = n \prod_{j=1}^r \left(1 - \frac{1}{p_j}\right).
\]
\end{block}
\end{frame}

\begin{frame}
\begin{block}{Proof}
Let $S = \{1, \dots, n\}$ and $A_j = \{p_j, 2p_j, 3p_j, \dots, n\}$ for $j=1, \dots, r$. Then
\begin{align*}
\phi(n) &= \left|S - (A_1 \cup \dots \cup A_r)\right| = \sum_i (-1)^i \sum_{j_1 < \dots < j_i} \left|A_{j_1} \cap \dots \cap A_{j_i}\right| \\
&= \sum_i (-1)^i \sum_{j_1 < \dots < j_i} \frac{n}{p_{j_1} \cdots p_{j_i}} \\
&= n \prod_{j=1}^r \left(1 - \frac{1}{p_j}\right).
\end{align*}
\end{block}
\end{frame}

\begin{frame}
\frametitle{Compositions with restrictions}

\begin{block}{Problem}
Count weak compositions of 7 with 7 parts none of which equals 2.
\end{block}

\begin{block}{Solution}
\begin{align*}
S &= \{x_1 + \dots + x_7 = 7, x_1, \dots, x_7 \geq 0\} \\
A_i &= \{(x_1, \dots, x_7) \in S \mid x_i = 2\} \\
|S - (A_1 \cup \dots \cup A_7)| &= ? \\
|S| &= |\{x_1 + \dots + x_7 = 7\}| = \binom{13}{6} \\
|A_i| &= |\{x_1 + \dots + x_6 = 5\}| = \binom{10}{5} \\
|A_i \cap A_j| &= |\{x_1 + \dots + x_5 = 3\}| = \binom{7}{4} \\
|A_i \cap A_j \cap A_k| &= |\{x_1 + \dots + x_4 = 1\}| = \binom{4}{3} \\
|S - (A_1 \cup \dots \cup A_7)| &= \binom{13}{6} - \binom{7}{1}\binom{10}{5} + \binom{7}{2}\binom{7}{4} - \binom{7}{3}\binom{4}{3} = 547
\end{align*}
\end{block}

\end{frame}

\begin{frame}
\frametitle{Inclusion-exclusion principle}
\begin{itemize}
    \item $|S - (A_1 \cup A_2)| = |S| - |A_1| - |A_2| + |A_1 \cap A_2|$
    \item $|S - (A_1 \cup A_2 \cup A_3)| = |S| - |A_1| - |A_2| - |A_3|$
    \hspace{2em} $+ |A_1 \cap A_2| + |A_2 \cap A_3| + |A_1 \cap A_3|$
    \hspace{2em} $- |A_1 \cap A_2 \cap A_3|$
\end{itemize}

\begin{block}{Theorem}
Let $S$ be a finite set, and let $A_1, \dots, A_r \subseteq S$. Then
\[ |S - (A_1 \cup \cdots \cup A_r)| = n_0 - n_1 + n_2 - n_3 + \cdots + (-1)^r n_r, \]
where
\begin{align*}
    n_0 &= |S|, \\
    n_i &= \sum_{1 \leq j_1 < \cdots < j_i \leq r} |A_{j_1} \cap \cdots \cap A_{j_i}| \quad (1 \leq i \leq r).
\end{align*}
\end{block}

\end{frame}

\begin{frame}
\begin{center}
Math 465: Introduction to Combinatorics
\end{center}

\vspace{0.5cm}

\begin{center}
Andrew Sack
\end{center}

\vspace{0.5cm}

\hrulefill

\vspace{0.2cm}

Homework \#5 will be due Monday evening.

Homework \#6 is posted.

\vspace{0.2cm}

\hrulefill

\vspace{0.2cm}

Exam \#1 will be held in class on Thursday, February 27.

A ``crib sheet" will be allowed. It can be 1 piece of paper up to

8.5"x11" and can be handwritten or typed.

\vspace{0.2cm}

\hrulefill

\vspace{0.2cm}

These slides will be posted on Canvas.
\end{frame}

\begin{frame}
\frametitle{Counting words using inclusion-exclusion}

\begin{block}{Problem}
How many words of length $n$ in the alphabet $\{1,2,3\}$ contain at least one 1, at least one 2, and at least one 3?
\end{block}

\begin{block}{Solution}
These words are in bijection with surjections $\{1, \dots, n\} \to \{1, 2, 3\}$.
Hence their number is $\sum_{i=0}^3 \binom{3}{i} (-1)^i (3-i)^n = 3^n - 3 \cdot 2^n + 3$.
\end{block}

\begin{block}{Problem}
How many anagrams of 111222333 do not contain three consecutive identical symbols?
\end{block}

\begin{block}{Solution}
Let $S = \{\text{anagrams of 111222333}\}$. Let $A_k$ consist of those words in $S$ that have the form $\dots kkk \dots$ ($k = 1,2,3$). By inclusion-exclusion, the answer is $\binom{9}{333} - 3\binom{7}{331} + 3\binom{5}{311} - \binom{3}{111} = 1314$.
\end{block}

\end{frame}

\begin{frame}
\frametitle{Euler's totient function}

\begin{block}{Definition}
Euler's totient function $\phi : \mathbb{Z}_{>0} \to \mathbb{Z}_{>0}$ is defined by
\[
\phi(n) = |\{k \in \mathbb{Z} \mid 1 \leq k \leq n, \gcd(k, n) = 1\}|.
\]
\end{block}

\begin{block}{Example}
\[
\phi(100) = |\{1, 3, 7, 9, 11, 13, 17, 19, \dots, 91, 93, 97, 99\}| = 40
\]
\end{block}

\begin{block}{Theorem [L. Euler]}
Let $\{p_1, \dots, p_r\}$ be the set of distinct prime divisors of $n$. Then
\[
\phi(n) = n \prod_{j=1}^{r} \left(1 - \frac{1}{p_j}\right).
\]
\end{block}

\begin{block}{Example}
\[
\phi(100) = 100 \cdot \frac{1}{2} \cdot \frac{4}{5} = 40
\]
\end{block}

\end{frame}

\begin{frame}
Solution, continued
\begin{align*}
|S| &= |\{x_1 + \cdots + x_4 = 10\}| = \binom{13}{3} \\
|A_1| &= |\{x_1 + \cdots + x_4 = 10, x_1 \geq 3\}| = |\{y_1 + \cdots + y_4 = 7\}| = \binom{10}{3} \\
|A_2| &= |\{x_1 + \cdots + x_4 = 10, x_2 \geq 4\}| = |\{y_1 + \cdots + y_4 = 6\}| = \binom{9}{3} \\
|A_3| &= |\{x_1 + \cdots + x_4 = 10, x_3 \geq 5\}| = |\{y_1 + \cdots + y_4 = 5\}| = \binom{8}{3} \\
|A_4| &= |\{x_1 + \cdots + x_4 = 10, x_4 \geq 6\}| = |\{y_1 + \cdots + y_4 = 4\}| = \binom{7}{3} \\
|A_1 \cap A_2| &= |\{\sum x_i = 10, x_1 \geq 3, x_2 \geq 4\}| = |\{\sum y_i = 3\}| = \binom{6}{3} \\
|A_1 \cap A_3| &= |\{\sum x_i = 10, x_1 \geq 3, x_3 \geq 5\}| = |\{\sum y_i = 2\}| = \binom{5}{3} \\
|A_1 \cap A_4| &= |\{\sum x_i = 10, x_1 \geq 3, x_4 \geq 6\}| = |\{\sum y_i = 1\}| = \binom{4}{3} \\
|A_2 \cap A_3| &= |\{\sum x_i = 10, x_2 \geq 4, x_3 \geq 5\}| = |\{\sum y_i = 1\}| = \binom{4}{3} \\
|A_2 \cap A_4| &= |\{\sum x_i = 10, x_2 \geq 4, x_4 \geq 6\}| = |\{\sum y_i = 0\}| = 1 \\
&\text{(all other intersections are empty)} \\
|S - (A_1 \cup \cdots \cup A_4)| &= \binom{13}{3} - \binom{10}{3} - \binom{9}{3} - \binom{8}{3} - \binom{7}{3} + \binom{6}{3} + \binom{5}{3} + 2\binom{4}{3} + 1 = 30
\end{align*}
\end{frame}

\begin{frame}
\frametitle{Derangements}

\begin{block}{Definition}
A \textbf{derangement} is a permutation $w = w_1 \cdots w_n \in S_n$ such that
\[
w_i \neq i \text{ for all } i.
\]
Let $d_n$ denote the number of derangements in $S_n$.
\end{block}

\begin{block}{Example}
\begin{tabular}{c|l|c}
$n = 1$ & & $d_1 = 0$ \\ \hline
$n = 2$ & 21 & $d_2 = 1$ \\ \hline
$n = 3$ & 231 \quad 312 & $d_3 = 2$ \\ \hline
$n = 4$ & 2143 \quad 2341 \quad 2413 & $d_4 = 9$ \\
& 3142 \quad 3412 \quad 3421 & \\
& 4123 \quad 4312 \quad 4321 & \\ \hline
$n = 5$ & 21453 \quad 21534 \quad 23154 \quad 23451 \quad 23514 & $d_5 = 44$ \\
& 24153 \quad 24513 \quad 24531 \quad 25134 \quad 25413 \quad 25431 & \\
& $\cdots \cdots \cdots \cdots \cdots \cdots \cdots \cdots \cdots \cdots \cdots \cdots$ &
\end{tabular}
\end{block}

\end{frame}

\begin{frame}
\frametitle{Counting permutations with prescribed ascents \& descents}
\begin{block}{Problem}
How many permutations $(a, b, c, d, e, f, g) \in S_7$ satisfy
$$a < b < c > d > e < f < g?$$
\end{block}
\begin{block}{Solution}
\begin{align*}
S &= \{(a, \dots, g) \in S_7 \mid a < b < c, \quad e < f < g\} \\
A_1 &= \{(a, \dots, g) \in S_7 \mid a < b < c < d, \quad e < f < g\} \\
A_2 &= \{(a, \dots, g) \in S_7 \mid a < b < c, \quad d < e < f < g\} \\
A_1 \cap A_2 &= \{(a, \dots, g) \in S_7 \mid a < b < c < d < e < f < g\} \\
|S - (A_1 \cup A_2)| &= \binom{7}{3} \binom{4}{3} - \binom{7}{4} - \binom{7}{3} + 1 = 71
\end{align*}
This method can be used to count permutations with any given pattern of ascents and descents.
\end{block}
\end{frame}

\begin{frame}
\frametitle{Recurrence for the number of derangements}

\begin{align*}
d_3 &= 6\left(1 - 1 + \frac{1}{2} - \frac{1}{6}\right) = 2 \\
d_4 &= 24\left(1 - 1 + \frac{1}{2} - \frac{1}{6} + \frac{1}{24}\right) = 9 \\
d_5 &= 120\left(1 - 1 + \frac{1}{2} - \frac{1}{6} + \frac{1}{24} - \frac{1}{120}\right) = 44
\end{align*}

\begin{block}{Proposition}
The derangement numbers $d_n$ satisfy the recurrence
\[
d_n = n d_{n-1} + (-1)^n.
\]
\end{block}

\begin{center}
\begin{tabular}{ |c|c|c|c|c|c|c|c|c|c| } 
 \hline
 $n$ & 1 & 2 & 3 & 4 & 5 & 6 & 7 & 8 & $\dots$ \\ 
 \hline
 $d_n$ & 0 & 1 & 2 & 9 & 44 & 265 & 1854 & 14833 & $\dots$ \\
 \hline
\end{tabular}
\end{center}

\end{frame}

\begin{frame}
\frametitle{Counting derangements}

\begin{block}{Theorem}
The number $d_n$ of derangements in $S_n$ is given by
\[d_n = n! \sum_{k=0}^n \frac{(-1)^k}{k!}.\]
\end{block}

\begin{block}{Proof}
Set $S = S_n$ and $A_i = \{w \in S_n \mid w_i = i\}$ for $i = 1, \dots, n$. Then
\begin{align*}
d_n &= |S - (A_1 \cup \dots \cup A_n)| \\
&= n! - n(n-1)! + \binom{n}{2}(n-2)! - \binom{n}{3}(n-3)! + \dots \\
&= \frac{n!}{0!} - \frac{n!}{1!} + \frac{n!}{2!} - \frac{n!}{3!} + \dots + (-1)^n \frac{n!}{n!} \\
&= n! \sum_{k=0}^n \frac{(-1)^k}{k!}.
\end{align*}
\end{block}

\end{frame}

\begin{frame}
    \frametitle{Derangements vs. all permutations}
    \[ d_n = n! \sum_{k=0}^{n} \frac{(-1)^k}{k!}. \]
\end{frame}

\begin{frame}
    \frametitle{Example}
    \begin{align*}
        d_3 &= 6 \left(1 - 1 + \frac{1}{2} - \frac{1}{6} \right) = 2 \\
        d_4 &= 24 \left(1 - 1 + \frac{1}{2} - \frac{1}{6} + \frac{1}{24}\right) = 9
    \end{align*}

    \[ \frac{1}{e} = e^{-1} = \frac{1}{0!} - \frac{1}{1!} + \frac{1}{2!} - \frac{1}{3!} + \frac{1}{4!} - \frac{1}{5!} + \cdots \]
\end{frame}

\begin{frame}
    \frametitle{Corollary}
    The number of derangements $d_n$ is the closest integer to $\frac{n!}{e}$.

    For example, for $n=4$, we have $\frac{4!}{e} \approx 8.83 \approx 9$, so $d_4 = 9$.

    Thus the probability that a uniformly random permutation is a derangement is approximately equal to $\frac{1}{e}$.
\end{frame}

\begin{frame}
\frametitle{Counting up-down permutations}
\begin{align*}
    E_7 &= |\{(a, b, c, d, e, f, g) \in S_7 \mid a <b>c<d>e< f > g\}| =? \\
    S &= \{(a, \dots, g) \in S_7 \mid a < b, c < d, e < f\} \\
    A_1 &= \{(a, \dots, g) \in S_7 \mid a < b < c < d, e < f\} \\
    A_2 &= \{(a, \dots, g) \in S_7 \mid a < b, c < d < e < f\} \\
    A_3 &= \{(a, \dots, g) \in S_7 \mid a < b, c < d, e < f < g\} \\
    A_1 \cap A_2 &= \{(a, \dots, g) \in S_7 \mid a < b < c < d < e < f\} \\
    A_2 \cap A_3 &= \{(a, \dots, g) \in S_7 \mid a < b, c < d < e < f < g\} \\
    A_1 \cap A_3 &= \{(a, \dots, g) \in S_7 \mid a < b < c < d, e < f < g\} \\
    A_1 \cap A_2 \cap A_3 &= \{(a, \dots, g) \in S_7 \mid a < b < c < d < e < f < g\} \\
    |S - (A_1 \cup A_2 \cup A_3)| &= \binom{7}{2}\binom{5}{2}\binom{3}{2} - 2\binom{7}{4}\binom{3}{2} - \binom{7}{2}\binom{5}{2} + \binom{7}{6} + \binom{7}{2} + \binom{7}{4} - 1 \\
    &= 272
\end{align*}
\end{frame}

\begin{frame}
\frametitle{Up-down permutations}

\begin{block}{Definition}
We say that $w = (w_1, \dots, w_n) \in S_n$ is an \textcolor{blue}{up-down permutation} if
\[w_1 < w_2 > w_3 < w_4 > \dots\]
We denote the number of up-down permutations in $S_n$ by $E_n$.
\end{block}

\begin{block}{Example}
\begin{tabular}{l|l|l}
$n=0$ & $\emptyset$ & $E_0 = 1$ \\ \hline
$n=1$ & $1$ & $E_1 = 1$ \\ \hline
$n=2$ & $12$ & $E_2 = 1$ \\ \hline
$n=3$ & $132 \ 231$ & $E_3 = 2$ \\ \hline
$n=4$ & $1324 \ 1423 \ 2314 \ 2413 \ 3412$ & $E_4 = 5$ \\ \hline
$n=5$ & $
\begin{array}{l}
13254 \ 14253 \ 14352 \ 15243 \ 15342 \ 23154 \\
24153 \ 24351 \ 25143 \ 25341 \ 34152 \ 34251 \\
35142 \ 35241 \ 45132 \ 45231
\end{array}
$ & $E_5 = 16$
\end{tabular}
\end{block}

\end{frame}

\begin{frame}
\frametitle{Recurrence for the number of up-down permutations}

\begin{center}
\begin{tabular}{ |c|c|c|c|c|c|c|c|c|c|c| }
\hline
$n$ & 0 & 1 & 2 & 3 & 4 & 5 & 6 & 7 & 8 & $\cdots$ \\
\hline
$E_n$ & 1 & 1 & 1 & 2 & 5 & 16 & 61 & 272 & 1385 & $\cdots$ \\
\hline
\end{tabular}
\end{center}

\begin{block}{Theorem/Exercise}
\begin{align*}
E_n &= \sum_{\substack{k+\ell=n-1 \\ k \text{ odd}}} \binom{n-1}{k} E_k E_\ell = \sum_{\substack{k+\ell=n-1 \\ k \text{ even}}} \binom{n-1}{k} E_k E_\ell \\
\sum_{n=0}^\infty \frac{E_n}{n!} x^n &= \sec(x) + \tan(x)
\end{align*}
\begin{align*}
E_6 &= \binom{5}{1} E_1 E_4 + \binom{5}{3} E_3 E_2 + \binom{5}{5} E_5 E_0 = 5 \cdot 1 \cdot 5 + 10 \cdot 2 \cdot 1 + 1 \cdot 16 \cdot 1 = 61 \\
E_7 &= \binom{6}{1} E_1 E_5 + \binom{6}{3} E_3 E_3 + \binom{6}{5} E_5 E_1 = 6 \cdot 1 \cdot 16 + 20 \cdot 2 \cdot 2 + 6 \cdot 16 \cdot 1 = 272
\end{align*}
\end{block}

\end{frame}
\end{document}